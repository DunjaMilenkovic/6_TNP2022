% !TEX encoding = UTF-8 Unicode

\documentclass[a4paper]{article}

\usepackage{color}
\usepackage{url}
\usepackage[T2A]{fontenc} % enable Cyrillic fonts
\usepackage[utf8]{inputenc} % make weird characters work
\usepackage{graphicx}

\usepackage{amsmath} % potrebno za matematiku


\usepackage[english,serbian]{babel}
%\usepackage[english,serbianc]{babel} %ukljuciti babel sa ovim opcijama, umesto gornjim, ukoliko se koristi cirilica

\usepackage[unicode]{hyperref}
\hypersetup{colorlinks,citecolor=green,filecolor=green,linkcolor=blue,urlcolor=blue}

%\newtheorem{primer}{Пример}[section] %ćirilični primer
\newtheorem{primer}{Primer}[section]

\begin{document}

\title{Diffie-Hellman algoritam\\ \small{Seminarski rad u okviru kursa\\Tehničko i naučno pisanje\\ Matematički fakultet}}

\author{Dunja Milenkovic\\ mi22056@alas.matf.bg.ac.rs\and Jana Vukovic\\mi22124@alas.matf.bg.ac.rs\and Lazar Nikolic\\mi22164@alas.matf.bg.ac.rs\and Sofija Janevska\\mi22035@alas.matf.bg.ac.rs }
\date{15.~novembar 2022.}
\maketitle

\abstract{
U ovom tekstu je ukratko prikazana osnovna forma seminarskog rada. Obratite pažnju da je pored ove .pdf datoteke, u prilogu i odgovarajuća .tex datoteka, kao i .bib datoteka korišćena za generisanje literature. Na prvoj strani seminarskog rada su naslov, apstrakt i sadržaj, i to sve mora da stane na prvu stranu! Kako bi Vaš seminarski zadovoljio standarde i očekivanja, koristite uputstva i materijale sa predavanja na temu pisanja seminarskih radova. Ovo je samo šablon koji se odnosi na fizički izgled seminarskog rada (šablon koji \emph{morate} da ispoštujete!) kao i par tehničkih pomoćnih uputstava. 

\tableofcontents

\newpage

\section{Uvod}
\label{sec:uvod}
Sve rasprostranjenija upotreba interneta širom sveta donosi sa sobom, kako značajne prednosti u načinu istraživanja, rada, povezivanja i sl. tako i povećan rizik od nebezbedne komunikacije i razmene podataka. Bilo da se radi o individualnim, privatnim razmenama informacija između ljudi ili rada nekih od najvećih državnih institucija, bezbedna i neometana interakcija i komunikacija među korisnicima na internetu je od presudnog značaja. Glavni problem nastaje kada spoljni korisnik koji nije predviđen kao deo veze pokuša da je prekine i dođe do određenih informacija koje su originalno bile namenjene jednom od povezanih korisnika. Upravo rešavanjem ovakvih problema bavi se \textbf{kriptografija}.

Kriptografija je oblast koja razvija tehnike koje omogućavaju zaštićenu i efikasnu digitalnu komunikaciju. Bez nje, komunikacija nebezbednim i nepoverljivim kanalima, što uključuje sve vrste mreža, a pogotovo internet, ne bi bila moguća. U ovom radu bavićemo se jednom od tehnika kriptografije koja se naziva \textbf{Diffie-Hellman algoritam} tj. \textbf{Diffie-Hellman protokol} i predstavićemo način rada algoritma, njegova osnovna svojstva, primene, prednosti i mane.

\section{Asimetrična kriptografija i osnove Diffie-Hellman algoritma}
Glavna klasifikacija kriptografskih algoritama deli ih na simetričnu kriptografiju, asimetričnu kriptografiju i heš funkcije; kako je Diffie-Hellman protokol predstavnik asimetrične kriptografije, nju ćemo posebno izdvojiti i definisati.

\textbf{Asimetrična kriptografija} ili tzv. tehnika javnog ključa podrazumeva korišćenje dva ključa – jednog za šivrovanje i drugog za dešifrovanje koda. „Tvorcima“ tj. pokretačima asimetrične kriptografije smatraju se \textbf{Whitfield Diffie} (Vitfild Difi) i Martin Hellman (Martin Helman), dok je \textbf{Ralph Merkle} (Ralf Merkl) takođe dao svoj doprinos ovoj oblasti radom na distribuciji javnog ključa. Njihova saradnja rezultirala je objavom zajedničkog rada pod nazivom "New Directions in Cryptography" ("Novi pravci u kriptografiji"), novembra 1976. godine. U njemu su opisane osnove Diffie-Hellman algoritma, koji je upravo otud i dobio naziv. Diffie i Hellman su 2015. godine dobili Tjuringovu nagradu za svoja dostignuća u oblasti kriptografije. 

U spomenutom radu, Diffie-Hellman algoritam opisan je kao algoritam koji koristi dva ključa i tako omogućava korisnicima bezbednu komunikaciju bez potrebe za deljenjem privatnog ključa. Osnovni princip na kome funkcioniše jeste korišćenje tzv. jednosmernih matematičkih funkcija, koje se lako izračunavaju u jednom smeru, dok isto ne važi i za njihove inverzne funkcije. Na ovaj način omogućeno je postojanje javnog ključa, bez rizika da ga treći korisnik koji nije predviđen kao deo mreže može zloupotrebiti



\section{Engleski termini i citiranje}	
\label{sec:termini_i_citiranje}

Na svakom mestu u tekstu naglasiti odakle tačno potiču informacije. Uz sve novouvedene termine u zagradi naglasiti od koje engleske reči termin potiče. 

Naredni primeri ilustruju način uvođenja enlegskih termina kao i citiranje.

\begin{primer}
Problem zaustavljanja (eng.~{\em halting problem}) je neodlučiv \cite{haltingproblem}.
\end{primer}

\begin{primer}
Za prevođenje programa napisanih u programskom jeziku C može se koristiti GCC kompajler \cite{gcc}.
\end{primer}

\begin{primer}
 Da bi se ispitivala ispravost softvera, najpre je potrebno precizno definisati njegovo ponašanje \cite{laski2009software}. 
\end{primer}

Ukoliko za unos referenci koriste datoteku {\em seminarski.bib},  prevođenje u pdf format u Linux okruženju može se uraditi na sledeći način:
\begin{verbatim}
pdflatex TemaImePrezime.tex 
bibtex TemaImePrezime.aux 
pdflatex TemaImePrezime.tex 
pdflatex TemaImePrezime.tex 
\end{verbatim}
Prvo latexovanje je neophodno da bi se generisao {\em .aux} fajl. {\em bibtex} proizvodi odgovarajući {\em .bbl} fajl koji se koristi za generisanje literature. 
Potrebna su dva prolaza (dva puta pdflatex) da bi se reference ubacile u tekst (tj da ne bi ostali znakovi pitanja umesto referenci). Dodavanjem novih referenci potrebno je ponoviti ceo postupak.  


Broj naslova i podnaslova je proizvoljan. Neophodni su samo Uvod i Zaključak. Na poglavlja unutar teksta referisati se po potrebi. 
\begin{primer}
U odeljku \ref{sec:naslov1} precizirani su osnovni pojmovi, dok su zaključci dati u odeljku \ref{sec:zakljucak}.
\end{primer}




\section{Slike i tabele}
\label{slike_i_tabele}

Slike i tabele treba da budu u svom okruženju, sa odgovarajućim naslovima, obeležene labelom da koje omogućava referenciranje. 

\begin{primer} Ovako se ubacuje slika. Obratiti pažnju da je dodato i 
\begin{verbatim}
\usepackage{graphicx}
\end{verbatim}

\begin{figure}[h!]
\begin{center}
\includegraphics[scale=0.2]{pande.png}
\end{center}
\caption{Pande}
\label{fig:pande}
\end{figure}

Na svaku sliku neophodno je referisati se negde u tekstu. Na primer, na slici \ref{fig:pande} prikazane su pande. 
\end{primer}

\begin{primer} I tabele treba da budu u svom okruženju, i na njih je neophodno referisati se u tekstu. Na primer, u tabeli \ref{tab:tabela1} su prikazana različita poravnanja u tabelama.

\begin{table}[h!]
\begin{center}
\caption{Razlčita poravnanja u okviru iste tabele ne treba koristiti jer su nepregledna.}
\begin{tabular}{|c|l|r|} \hline
centralno poravnanje& levo poravnanje& desno poravnanje\\ \hline
a &b&c\\ \hline
d &e&f\\ \hline
\end{tabular}
\label{tab:tabela1}
\end{center}
\end{table}

\end{primer}





\section{Algoritam}
\label{sec:naslov1}

Pre početka razmene ključeva, uspostavljaju se 2 javno poznata broja: $p$ i $g$.
\begin{itemize}

    \item $p$ - bezbedan prost broj i modulo po kom radimo, preporuka je da to bude broj dužine 3072 bita (broj reda $10^{925}$)[ref needed]
    \item $g$ - generator, mora da bude primitivni koren od $p$

\end{itemize}

\emph{Bezbedan prost broj}: $p$ je bezbedno prost ako može da se izrazi kao $p = 2q + 1$ gde je $q$ takođe prost broj. $q$ je Sofija Žermen prost (eng. \textit{Sophie Germain prime}). Ovo je bitno kako bi se izbegao specifičan napad na Diffie-Hellman --- Silver-Polig-Helman algoritam (eng. \emph{Silver-Pohlig-Hellman algorithm}) \cite{pohlig-hellman}

\emph{Primitivni koren}: $g$ je primitivni koren po modulu $p$ ako za svaki celi broj $a$ koji je uzajamno prost sa $p$ postoji celi broj $k$ tako da je
    \[g^k \equiv a \mod p\]
Takvo $k$ se naziva indeks ili diskretni logaritam od $a$ sa odnovom $g \mod p$.



\subsection{Koraci algoritma}

Recimo da Anastasija i Boban žele da razmene ključeve.
%TREBA POPRAVITI OVE MODULE BAŠ SU RUŽNI !!!
\begin{itemize}
    \item Pretpostavljamo da su se Anastasija i Boban već dogovorili oko brojeva $g$ i $p$
    \item Anastasija nasumično bira neki tajni broj $a < p$, ovo je njen privatni ključ
    \item Zatim kalkuliše njen javni ključ koji je jednak $A = g^a \mod p$ 
    \item Boban će uraditi isto, nakon čega poseduje privatni ključ $b < p$ i javni ključ $B = g^b \mod p$
    \item Anastasija i Boban sada razmene njihove javne ključeve (privatni ključevi ostaju tajni)
    \item Anastasija sada poseduje svoje ključeve, kao i Bobanov javni ključ
    \item Ona će izračunati tajni broj $K = B^a \mod p$. Ako zamenimo $B$, videćemo da je
        \[K = (g^b \mod p)^a \mod p = (g^b)^a \mod p = g^{ab} \mod p\]
    \item Boban poseduje svoje ključeve i Anastasijin javni ključ
    \item On će istim postupkom kao i Anastasija pronaći $K$
        \[K = A^b \mod p = (g^a \mod p)^b \mod p = (g^a)^b \mod p = g^{ab} \mod p\]
\end{itemize}

Primetimo da će i Anastasija i Boban izračunati isti broj $K$. To $K$ predstavlja ključ za enkripciju ključeva (engl. \emph{Key-Encryption Key (KEK)}). Pomoću tog ključa mogu sinhrono da razmene novi ključ za enkripciju sadržaja (engl. \emph{Content-Encryption Key (CEK)}) i da nastave komunikaciju pomoću njega.

Neko ko je prisluškivao ovoj razmeni zna $p$, $g$, $g^a$ i $g^b$. Da bi od ovih brojeva pronašao $K$, on mora da izračuna vrednost $a$ ili $b$, problem koji se zove \emph{komputacioni Diffie-Hellman problem} (engl. \emph{computational Diffie-Hellman problem (CDH)}). 




\subsection{Prvi podnaslov}
\label{subsec:podnaslov1}

Ovde pišem tekst. 
Ovde pišem tekst. 
Ovde pišem tekst. 
Ovde pišem tekst. 
Ovde pišem tekst. 
Ovde pišem tekst. 
Ovde pišem tekst. 

\subsection{Drugi podnaslov}
\label{subsec:podnaslov2}

Ovde pišem tekst. 
Ovde pišem tekst. 
Ovde pišem tekst. 
Ovde pišem tekst. 
Ovde pišem tekst. 
Ovde pišem tekst. 

\section{Drugi naslov}
\label{sec:naslov2}

Ovde pišem tekst. 
Ovde pišem tekst. 
Ovde pišem tekst. 
Ovde pišem tekst. 

\subsection{... podnaslov}
\label{subsec:podnaslovN}

Ovde pišem tekst. 
Ovde pišem tekst. 
Ovde pišem tekst. 
Ovde pišem tekst. 
Ovde pišem tekst. 
Ovde pišem tekst. 

\section{n-ti naslov}
\label{sec:naslovN}

Ovde pišem tekst. 
Ovde pišem tekst. 
Ovde pišem tekst. 
Ovde pišem tekst. 
Ovde pišem tekst. 

\subsection{... podnaslov}
\label{subsec:podnaslovK}

Ovde pišem tekst. 
Ovde pišem tekst. 
Ovde pišem tekst. 
Ovde pišem tekst. 
Ovde pišem tekst. 

\subsection{... podnaslov}
\label{subsec:podnaslovM}

Ovde pišem tekst. 
Ovde pišem tekst. 
Ovde pišem tekst. 
Ovde pišem tekst. 
Ovde pišem tekst. 

\section{Poslednji naslov}
\label{sec:naslovM}

Ovde pišem tekst. 
Ovde pišem tekst. 
Ovde pišem tekst. 
Ovde pišem tekst. 
Ovde pišem tekst. 
Ovde pišem tekst. 
Ovde pišem tekst. 
Ovde pišem tekst. 
Ovde pišem tekst. 

\section{Zaključak}
\label{sec:zakljucak}

Ovde pišem zaključak. 
Ovde pišem zaključak. 
Ovde pišem zaključak. 
Ovde pišem zaključak. 
Ovde pišem zaključak. 
Ovde pišem zaključak. 
Ovde pišem zaključak. 
Ovde pišem zaključak. 
Ovde pišem zaključak. 
Ovde pišem zaključak. 
Ovde pišem zaključak. 
Ovde pišem zaključak. 


\addcontentsline{toc}{section}{Literatura}
\appendix

\iffalse
\bibliography{seminarski} 
\bibliographystyle{plain}
\fi

\begin{thebibliography}{9}

\bibitem{kriptografija} Gary C. Kessler. \emph{An Overview of Cryptography}, 2015.

\bibitem{Diffie-Hellman protokol i primene} Maryam Ahmed, Baharan Sanjabi, Difo Aldiaz, Amirhossein Rezaei, Habeeb Omotunde. \emph{Diffie-Hellman and Its Application in Security Protocols}, International Journal of Engineering Science and Innovative Technology (IJESIT), 2012.

\bibitem{haltingproblem} A. M. Turing. \emph{On Computable Numbers, with an application to the Entscheidungsproblem}. Proceedings of the London Mathematical Society, 2(42):230–265, 1936.

\bibitem{pohlig-hellman} Mollin, Richard (2006-09-18). \emph{An Introduction To Cryptography} (2nd ed.). Chapman and Hall/CRC. p. 344

\end{thebibliography}


\appendix
\section{Dodatak}
Ovde pišem dodatne stvari, ukoliko za time ima potrebe.
Ovde pišem dodatne stvari, ukoliko za time ima potrebe.
Ovde pišem dodatne stvari, ukoliko za time ima potrebe.
Ovde pišem dodatne stvari, ukoliko za time ima potrebe.
Ovde pišem dodatne stvari, ukoliko za time ima potrebe.

    
\end{document}
