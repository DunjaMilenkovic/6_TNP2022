\documentclass[14pt]{beamer}
\usepackage{beamerthemeshadow}
\usepackage{graphicx}
\usepackage{color}
\usepackage[utf8]{inputenc}
\usepackage{hyperref}
\usepackage{makecell}                       %potrebno za formatiranje tabele
%\usepackage[flushleft]{threeparttable}     %ne koristimo ovo?

\usepackage[english, serbian]{babel}        %da ne bi pisalo "Table" nego "Tabela" i slično

\definecolor{babyblueeyes}{rgb}{0.63, 0.79, 0.95}
\setbeamercolor{structure}{fg=babyblueeyes}

\def\d{{\fontencoding{T1}\selectfont\dj}}
\def\D{{\fontencoding{T1}\selectfont\DJ}}


\title{Diffie-Hellman algoritam}
\subtitle{-- Tehničko i naučno pisanje --}
\author{Jana Vuković, Sofija Janevska, Lazar Nikolić, Dunja Milenković}
\institute{Matematički fakultet\\Univerzitet u Beogradu}
\date{
	\footnotesize{Beograd, 2022.}	
}

\begin{document}
\begin{frame}
	\thispagestyle{empty}
	\titlepage
\end{frame}

\addtocounter{framenumber}{-1}

\section{Uvod}

\begin{frame}[fragile]\frametitle{Slajd 1}
	\begin{itemize}
		\item ...
	\end{itemize}
\end{frame}

\section*{}         % ovo kreira "fantom" poglavlje

\begin{frame}[fragile]\frametitle{Slajd 2}
	 \begin{itemize}
		\item ...
	\end{itemize}
\end{frame}

\begin{frame}[fragile]\frametitle{Slajd 3 - Koraci Algoritma}
    \begin{table}[h]
        \begin{center}
        \resizebox{10cm}{!}{
            \begin{tabular}{ |c|c|c|c| }
                \hline
                Trenutni korak & Anastasija zna & Javno poznato & Boban zna \\
                \hline
                Početak algoritma &  & $g, p$ &  \\
                \hline
                \makecell{Svako računa\\svoj ključ} & $a, A=g^a \mod p $ & $g, p$ & $b, B=g^b \mod p$ \\
                \hline
                \makecell{Razmena\\javnih ključeva} & $a, A, B$ & $g, p, A, B$ & $b, B, A $ \\
                \hline
                \makecell{Svako računa isto, \\ tajno K} &
                \makecell{{$\!\begin{aligned}
                    a, A, B,\\K=B^a \mod p\\=g^{ab} \mod p
                \end{aligned}$} } &
                \makecell{$g, p, A, B$} &
                \makecell{{$\!\begin{aligned}
                b, B, A,\\K=A^b \mod p\\=g^{ab} \mod p
                \end{aligned}$}\\}\\
                \hline
            \end{tabular}
        }
        \caption{Tajnost promenljivih u toku algoritma}
        \end{center}
    \end{table}

\end{frame}

\begin{frame}[fragile]\frametitle{Slajd 4 - Problemi oslonci}
    \small
    \emph{Sve u cikličnoj grupi $G$ reda $q$}
	\begin{itemize}	
		\item Problem diskretnog logaritma
        \item[] U cikličnoj grupi $G$ reda $q$, pronalaženje $k$, $0 \le k \le q - 1$ tako da $x = g^a$ \pause
        \item Komputacioni Diffie-Hellman
        \item[] $a, b \in \mathbb{Z}\setminus q\mathbb{Z}, A = g^a, B = g^b$. Pronalaženje $g^{ab}$ ako znamo $A,B$ \pause
        \item Odlučujući Diffie-Hellman
        \item[] $a, b, c \in \mathbb{Z}\setminus q\mathbb{Z}, A=g^a, B=g^b, C=g^c\ ili\ C=g^{ab}$
	\end{itemize}
\end{frame}

\begin{frame}[fragile]\frametitle{Slajd 5}
	\begin{itemize}	
		\item ...
	\end{itemize}
\end{frame}

\begin{frame}[fragile]\frametitle{Slajd 6}
	\begin{itemize}	
		\item ...
	\end{itemize}
\end{frame}

\begin{frame}[fragile]\frametitle{Slajd 7}
	\begin{itemize}	
		\item ...
	\end{itemize}
\end{frame}

\section{Zaključak}

\begin{frame}[fragile]\frametitle{Slajd 8}
	\begin{itemize}	
		\item ...
	\end{itemize}
\end{frame}

\end{document}
