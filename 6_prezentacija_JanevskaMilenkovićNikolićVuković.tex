\documentclass[14pt]{beamer}
\usepackage{beamerthemeshadow}
\usepackage{graphicx}
\usepackage{color}
\usepackage[utf8]{inputenc}
\usepackage{hyperref}
\usepackage{makecell}                       %potrebno za formatiranje tabele
%\usepackage[flushleft]{threeparttable}     %ne koristimo ovo?

\usepackage[english, serbian]{babel}        %da ne bi pisalo "Table" nego "Tabela" i slično

\definecolor{babyblueeyes}{rgb}{0.63, 0.79, 0.95}
\setbeamercolor{structure}{fg=babyblueeyes}

\def\d{{\fontencoding{T1}\selectfont\dj}}
\def\D{{\fontencoding{T1}\selectfont\DJ}}


\title{Diffie-Hellman algoritam}
\subtitle{-- Tehničko i naučno pisanje --}
\author{Jana Vuković, Sofija Janevska, Lazar Nikolić, Dunja Milenković}
\institute{Matematički fakultet\\Univerzitet u Beogradu}
\date{
	\footnotesize{Beograd, 2022.}	
}

\begin{document}
\begin{frame}
	\thispagestyle{empty}
	\titlepage
\end{frame}

\addtocounter{framenumber}{-1}

\section{Algoritam}

\begin{frame}[fragile]\frametitle{Osnove Diffie-Hellman algoritma}
	\begin{itemize}
		\item Kriptografija javnog ključa
		\item Vitfild Difi i Martin Helman - "Novi pravci u kriptografiji"
		\item Dva ključa
		\item Jednosmerne matematičke funkcije
	\end{itemize}
\end{frame}

%\section*{}    %ovo kreira "fantom" poglavlje

\begin{frame}[fragile]\frametitle{Koraci Algoritma}
    \begin{table}[h]
        \begin{center}
        \resizebox{10cm}{!}{
            \begin{tabular}{ |c|c|c|c| }
                \hline
                Trenutni korak & Anastasija zna & Javno poznato & Boban zna \\
                \hline
                Početak algoritma &  & $g, p$ &  \\
                \hline
                \makecell{Svako računa\\svoj ključ} & $a, A=g^a \mod p $ & $g, p$ & $b, B=g^b \mod p$ \\
                \hline
                \makecell{Razmena\\javnih ključeva} & $a, A, B$ & $g, p, A, B$ & $b, B, A $ \\
                \hline
                \makecell{Svako računa isto, \\ tajno K} &
                \makecell{{$\!\begin{aligned}
                    a, A, B,\\K=B^a \mod p\\=g^{ab} \mod p
                \end{aligned}$} } &
                \makecell{$g, p, A, B$} &
                \makecell{{$\!\begin{aligned}
                b, B, A,\\K=A^b \mod p\\=g^{ab} \mod p
                \end{aligned}$}\\}\\
                \hline
            \end{tabular}
        }
        \caption{Šematski prikaz razmena promenljivih tokom algoritma}
        \end{center}
    \end{table}

\end{frame}

\begin{frame}[fragile]\frametitle{Problemi oslonci}
    \small
    \emph{Sve u cikličnoj grupi $G$ reda $q$}
	\begin{itemize}	
		\item Problem diskretnog logaritma
        \item[] Pronalaženje $k$, $0 \le k \le q - 1$ tako da $x = g^k$ %\pause
        \item Komputacioni Diffie-Hellman
        \item[] $a, b \in \mathbb{Z}\setminus q\mathbb{Z}, A = g^a, B = g^b$. Pronalaženje $g^{ab}$ ako znamo $A,B$ %\pause
        \item Odlučujući Diffie-Hellman
        \item[] $a, b, c \in \mathbb{Z}\setminus q\mathbb{Z}, A=g^a, B=g^b, C=g^c\ ili\ C=g^{ab}$
	\end{itemize}
\end{frame}

\begin{frame}[fragile]\frametitle{Primer Diffie-Hellman algoritma}
        \small
	\begin{itemize}	
		\item Neka je izabrani prost broj $q=353$. Prost koren za ovu vrednost je $\alpha=3$.
            \item Tajni ključevi: $X_A=97$, $X_B=233$.
            \item Javni ključevi: $Y_A=3^{97}\mod353=40$, \newline $Y_B=3^{233}\mod353=248$.
            \item Razmena javnih ključeva $Y_A$ i $Y_B$; Izračunavanje tajnog ključa $K$.
            \item Anastasija tajni ključ izračunava po formuli: $$K=(Y_B)^{X_A}\mod353=248^{97}\mod353=160$$
            \item Boban tajni ključ izračunava po formuli: $$K=(Y_A)^{X_B}\mod353=40^{233}\mod353=160$$
	\end{itemize}
\end{frame}

\begin{frame}[fragile]\frametitle{ElGamal}
        \footnotesize
	\begin{itemize}	
        \itemsep0em
		\item Zasnovan je na kompleksnosti izračunavanja vrednosti diskretnih logaritama.
            \item Sastoji se iz tri glavne tačke - generisanje ključa, enkripcija i dekripcija.
                    \item $q=19$, $\alpha=10$. 
                    \item $X_A=5$, $Y_A=\alpha^{X_A}\mod q=10^5\mod 19=3$.
                    \item $M=17$, $k=6$.
                    \item $K=Y_A^k\mod q=3^6\mod 19=7$.
                    \item $C1=\alpha^k \mod q=10^6\mod 19=5$.
                     \item $C2=KM\mod q=7\cdot17\mod 19=5$
                     \item $K=C_1^{X_A}\mod q=11^5\mod 19$
                    \item $7\cdot K^{-1} \equiv 1 \mod 19 \implies K^{-1}=11$
                    \item $M=(C_2K^{-1})\mod q=5\cdot 11 \mod 19=17$                
	\end{itemize}
\end{frame}

\begin{frame}[fragile]\frametitle{Napadi i primena}
\section{Napadi i primena}
	\begin{itemize}	
        \item Čovek u sredini 
        
        $g^{a’}$ , $g^{b’}$ , $ENC_{g^{ab’}} (m)$ , $ENC_{g^{a’b}} (m’)$
	\item Autsajder (eng. $Outsider$) napad
        \item Insider napad

        $g^{a} = 1$ tj. $g^{ab} = 1$ ; $g^a = g$
        \item DoS (eng. $Denial$ $of$ $Service$)
        \item Primene na bezbednosnim protokolima (SSL, SSH, IPSec)
	\end{itemize}
\end{frame}

\begin{frame}[fragile]\frametitle{Literatura}
	\begin{itemize}	
		\scriptsize \item Maryam Ahmed, Baharan Sanjabi, et al. Diffie-Hellman and Its Application in Security Protocols, (IJESIT), 2012.
		\scriptsize \item  E. Rescorla (June 1999) Diffie-Hellman Key Agreement Method
		\scriptsize \item  Kevin S. McCurley (1990). The Discrete Logarithm Problem
		\scriptsize \item  A. Joux, K. Nguyen (2003). Separating Decision Diffie-Hellman from
Computational Diffie-Hellman in Cryptographic Groups
        \scriptsize \item  J. F. Raymond, A. Stiglic (2000). Security Issues in the Diffie-Hellman
Key Agreement Protocol
        \scriptsize \item  T.Elgamal (1985). A public key cryptosystem and a signature scheme
based on discrete logarithms
	\end{itemize}
\end{frame}

\end{document}
